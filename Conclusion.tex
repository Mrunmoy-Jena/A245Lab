The experiment was a very interesting way to familiarize oneself with non linear optics.
Firstly the diode laser characteristics were measured and it was seen that it starts producing power only after a threshold value of the injection current. Also, the calibration of removable attenuator and then a variable attenuator was done.

After adjustment of various optical elements, it was possible to guide the beam into the non linear crystal to produce second harmonic waves. Quadratic dependence of SH beam intensity on fundamental power and a $\cos^{4}$ like dependence on fundamental wave polarization were found, indicating that Type I phase matching was attained. Further, a sinc like dependence on crystal temperature were observed.

Two different methods: one using a diffraction grating and other with a Michelson interferometer were used to find the wavelength ratio. From both these observation we were able to verify the 2:1 ratio between SH and fundamental wavelengths.